\chapter{TINJAUAN PUSTAKA}

\section{Penelitian Terdahulu}
Penelitian terdahulu merupakan hasil telusuran tentang kepustakaan yang mengupas topik penelitian yang relevan dengan penelitian yang akan diteliti. Hal ini merupakan bukti pendukung bahwa topik atau materi yang diteliti memang merupakan suatu permasalahan yang penting karena juga merupakan concern banyak orang, sebagaimana ditunjukkan oleh kepustakaan yang dirujuk. Penelitian terdahulu juga dapat menunjukkan posisi penelitian yang dilakukan di antara penelitian yang telah ada (\textit{state of the art}) sehingga dapat menunjukkan kebaruan (\textit{novelty}) penelitian. Penelitian terdahulu dapat bersumber dari skripsi, jurnal, prosiding, dll.

\section{Kajian Teori}
Pada bagian ini dinyatakan berbagai teori yang berkaitan dengan topik penelitian. Pada bab ini pula dimungkinkan diajukan lebih dari satu teori atau data sekunder/tersier untuk membahas permasalahan yang menjadi topik skripsi, sepanjang teori–teori dan/atau data sekunder/tersier itu berkaitan.


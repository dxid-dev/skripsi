\chapter{PENDAHULUAN}
\pagenumbering{arabic}

\section{Latar Belakang}
Latar belakang penelitian mengungkapkan keingintahuan mahasiswa tentang fenomena/gejala yang menarik untuk diteliti dengan menunjukkan signifikansi penelitian bagi pengembangan pengetahuan ilmiah. Dari pihak peneliti, pengungkapan bagian ini dapat didasarkan atas pertanyaan-pertanyaan berikut: 
\begin{enumerate}
  \item Tentang topik yang diteliti, apa-apa saja informasi yang telah diketahui, baik teoretis maupun faktual,
  \item Berdasarkan informasi yang diperoleh, adakah ditemukan adanya permasalahan baru bukan meneliti atau meniru masalah yang sudah ada,
  \item Dari permasalahan yang dapat diidentifikasi, bagian mana yang menarik untuk diteliti,
  \item Apakah mungkin secara teoretis dan teknis masalah itu diteliti,
  \item Latar Belakang harus mengarah ke identifikasi masalah.
\end{enumerate}

\section{Identifikasi Masalah}
Identifikasi masalah adalah inti fenomena yang akan diteliti sebagai akibat adanya kesenjangan teori dan realitas. Identifikasi masalah dinyatakan dalam wujud kalimat tanya yang dilengkapi dengan kata tanya; apa dan bagaimana. Misalnya:
\begin{enumerate}
  \item Apa yang hendak dibahas?
  \item Bagaimana topik tersebut ditampilkan?
\end{enumerate}

\section{Batasan Masalah}
Bagian ini menjadi salah satu bagian penting dalam Pendahuluan. Setelah paparan Latar Belakang, maka masalah yang diangkat pada pekerjaan penelitian perlu dirumuskan dengan baik. Perumusan ini sebaiknya dibahasakan tidak dalam bentuk kalimat pertanyaan, melainkan kalimat aktif, dan dapat memuat lebih dari satu rumusan. Sejalan dengan ini, setiap masalah yang diangkat selalu memiliki batas. Ada batasan, asumsi, atau kriteria yang menjadi pembatas atas masalah yang diangkat dalam penelitian TA, sehingga arah penelitian dapat fokus. Batasan ini perlu dituliskan secara tegas, dan dapat saja memuat lebih dari satu.

\section{Tujuan Penelitian}
Tujuan penelitian mengungkapkan arah dan tujuan umum apa yang akan dicapai dalam penelitian. Tujuan penelitian mengetengahkan indikator-indikator/ aspek- aspek yang hendak ditemukan dalam penelitian, terutama berkaitan dengan variabel-variabel yang akan diteliti (ditandai dengan verba yang mengindikasikan hasil; memetakan, mengklasifikasi, menunjukkan, mendeskripsikan). Tujuan Penelitian umumnya diungkapkan dalam wujud kalimat deskriptif dari identifikasi masalah.
\begin{enumerate}
  \item Mendeskripsikan topik A,
  \item Memetakan topik A.
\end{enumerate}
